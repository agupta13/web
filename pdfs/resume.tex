% LaTeX file for resume
% This file uses the resume document class (res.cls)
\documentclass[11pt]{res}
%\usepackage{helvetica} % uses helvetica postscript font (download helvetica.sty)
%\usepackage{newcent} % uses new century schoolbook postscript font
\newsectionwidth{0pt} % So the text is not indented under section headings
\usepackage{fancyhdr} % use this package to get a 2 line header
\renewcommand{\headrulewidth}{0pt} % suppress line drawn by default by fancyhdr
\setlength{\headheight}{12pt} % allow room for 2-line header
\setlength{\headsep}{12pt} % space between header and text
\setlength{\headheight}{24pt} % allow room for 2-line header
\pagestyle{fancy} % set pagestyle for document
%\rhead{ {\it Arpit Gupta}\\{\it p. \thepage} } % put text in header (right side)
\cfoot{} % the foot is empty
\topmargin=-0.75in % start text higher on the page
% \usepackage{simplemargins}
%\setbottommargin{0in}
\addtolength{\textheight}{1.6in}
\addtolength{\oddsidemargin}{-0.25in}
\addtolength{\evensidemargin}{-0.25in}
\addtolength{\textwidth}{0.25in}
%\leftmargin=-2in
\begin{document}
\thispagestyle{empty} % this page has no header
\name{{\huge \sc{Arpit Gupta}}\\[10pt]}% the \\[12pt] adds a blank line after name
\address{3337 Klaus\\
Atlanta, 30308 \\ }
\address{e-mail: agupta80@gatech.edu \\
Website : cc.gatech.edu/agupta80}
\begin{resume}
\section{Research Interest}
My research interest broadly is in Systems and Networks and specific focus on intersection of 
Internet Routing and Software Defined Networks (SDN). 
%Philosophically, my goal is to make Internet routing robust and flexible. 
\section{Education}
{\sl PhD}, Computer Science \hfill Fall 2013-Present\\
Georgia Tech, Atlanta, USA \hspace{0.2in}


{\sl Masters of Science}, Computer Science \hfill Spring 2013\\ 
North Carolina State University, Raleigh, USA \hspace{0.2in}


{\sl Bachelor of Technology}, ECE \hfill Spring 2009\\
Indian Institute of Technology (IIT), Roorkee, India \hspace{0.2in}

%\\ Adviser: Dr. R. Mitra
\section{{Honors and Awards}}
\begin{itemize} \itemsep -2pt
\item Awarded \textbf{Internet-2 Innovation Award}, Fall 2013.
\item Awarded \textbf{Meissner Fellowship}, Purdue University for Fall 2013.
\item WiFox work was Slashdotted and covered by CBS, Engadget, TechCrunch, Telegraph etc.
%\item \textbf{SWE Intern at Google Inc.} for Summer 2011.
%\item Awarded NSF Travel Grant for CoNEXT 2010, Philadelphia, USA.
\item Awarded \textbf {College of Engineering Fellowship} at NC State University for 2010-11.
\item Won \textbf{Motorola Innovation Award}, Fall 2007.
\item Ranked Top 99.57 percentile (787) among 170,000 IIT-JEE-2005 aspirants.
\item Ranked Top 99.88 percentile (591) among 450,000 AIEEE-2005 aspirants.
%\item State Rank 30 in AIEEE-2005 among 100,000 candidates.
%\item Selected for NATIONAL STANDARD EXAMINATION IN CHEMISTRY-2005 among 250,000 aspirants.
\end{itemize}

\section{{Selected Publications}}
%\begin{bibsection}
\begin{itemize} \itemsep -2pt
\item \textit {\textbf{Arpit Gupta}, Laurent Vanbever, Muhammad Shahbaz, Sean P. Donovan, Brandon Schlinker, Nick Feamster, Jennifer Rexford, Scott Shenker, Russ Clark, Ethan Katz-Bassett}, "SDX: A Software Defined Internet Exchange", ACM SIGCOMM 2014, Chicago.
\item \textit {\textbf{Arpit Gupta}, Matt Calder, Nick Feamster, Marshini Chetty, Enrico Calandro, Ethan Katz-Bassett}, "Peering at the Internet’s Frontier: A First Look at ISP Interconnectivity in Africa", PAM 2014, Los Angeles.
\item \textit {Hyojoon Kim, \textbf{Arpit Gupta}, Muhammad Shahbaz, Joshua Reich, Nick Feamster, Russ Clark}, "Simpler Network Configuration with State-Based Network Policies", Tech Report, GT-CS-13-04.
\item \textit {\textbf{Arpit Gupta}, Jeongki Min, Injong Rhee}, "WiFox: Scaling WiFi Performance for Large Audience Environments", ACM SIGCOMM CoNEXT, 2012, Nice.

%\item \textit {\textbf{Arpit Gupta}, Abhishek Kr. Gupta, Cosmin Bocaniala, Venkat Sastry}, “Avoidance of threat zone by UAV for automated navigation”, in proc. of \emph {IEEE- INDICON}-2008, Kanpur, India, 2008

\end{itemize}

\section{{Talks}}
\begin{itemize} \itemsep -2pt
\item SDX: A Software Defined Internet Exchange
\begin{itemize} 
\item Stanford NetSeminar, Fall 2014
\item ACM SIGCOMM, Fall 2014
\item Facebook Inc., Menlo Park, Summer 2014
\item Microsoft, Palo Alto, Summer 2014
\item GENI 21, Davis, Summer 2014
\item NANOG 59, Phoenix, Fall 2013
\end{itemize}
\end{itemize}

\newpage

\section{{Talks}}
\begin{itemize} \itemsep -2pt
\item Peering at the Internet's Frontier
\begin{itemize} 
\item PAM, Los Angeles, Spring 2014
\end{itemize}

\item WiFox: Scaling WiFi Performance for Large Audience
\begin{itemize} 
\item SyNRG, Duke University, Fall 2012
\item SysChat, UNC Chapel Hill, Fall 2012
\item ACM SIGCOMM CoNEXT, Nice, December 2012
\end{itemize}

\end{itemize}




\section{{Research Experience}}

{\sl Research Student: } \textbf{Georgia Tech, USA} \hfill June 2013- Present
\begin{itemize} \itemsep -2pt % reduce space between items
\item Adviser: Nick Feamster
\item Developed Software Defined Internet exchange point (SDX). 
\item Analyzed ISP inter-connectivity in developing regions.
\item Worked on Kinetic, simplifying network configuration and management. 

\end{itemize}
{\sl Research Student: } \textbf{North Carolina State University, USA} \hfill September 2010-May 2013
\begin{itemize} \itemsep -2pt % reduce space between items
\item Adviser: Injong Rhee
\item Developed WiFox, solving WiFi peformance degradation for large conference environments.
%\item Defining problem statement for Proxy Assisted Mobile Data Offloading.
\item WiFi scan interval optimization problem to minimize energy wasted due to unnecessary WiFi scans.
\end{itemize}



{\sl Summer Intern: } \textbf{Google Inc. Mountain View, CA, USA} \hfill May-August 2011 
\begin{itemize} \itemsep -2pt % reduce space between items
\item Adviser: Nandita Dukkipati
\item Performance of short flows, serving Google's search traffic is very critical. 
To study the impact of TCP timeouts on performance of these short flows, I ran some experiments over 
Google's search traffic to quantify this impact. 
%Studied the implementation of TCP timeouts to identify possible problems for short flows. 
\item To study the impact of tcp timeouts on short flows, instrumented kernel changes collecting performance metrics and ran 
experiments over Google's front-end server traffic. 
\end{itemize}


{\sl Project Assistant: } \textbf{Indian Institute of Science, Bangalore, India} \hfill March-August 2010
\begin{itemize} \itemsep -2pt % reduce space between items
\item Adviser: Dr. Anurag Kumar
%\item Aim: Providing proportional fairness to mobile stations with different link quality in 802.11 WLAN Networks.
\item Developed a scheduling algorithm ensuring fairness to clients with disparate link qualities. 
%Implemented the scheduler in C language at a server and simulated the behavior on QUALNET
\end{itemize}

\section{{Teaching Experience}}

{\sl Teaching Assistant: } \textbf{Georgia Tech, USA} \hfill Summer 2014 \& Fall 2014
\begin{itemize} \itemsep -2pt % reduce space between items
\item Instructor: Nick Feamster
\item TA for online SDN Coursera course (50K+ students) 
\item TA for in-class SDN Lab course (30 students)
\end{itemize}

{\sl Teaching Assistant: } \textbf{North Carolina State University, USA} \hfill Fall 2012
\begin{itemize} \itemsep -2pt % reduce space between items
\item Instructor: Rudra Dutta
\item TA for in-class computer networks course (100 students)
\end{itemize}




%\section{{Patents}}
%\begin{itemize} \itemsep -2pt
%\item Methods and Apparatus for ADWISER: An Integrated Approach for Internet Access Bandwidth and
%Performance Management of an Enterprise Network, \emph{Indian patent}, filed in August 2010.
%%\item Methods and Apparatus for Direct WiFi: Sensing WiFi APs presence using cellular signals, application filed in August 2011.
%\end{itemize}
%%\end{bibsection}
%%\newpage
%\section{{Relevant Courses }}
%\vspace{5pt}
%\begin{tabular} {l l l l }
%& & Data Structures and Algorithms &	Operating Systems Principles \\
%& & Automated Learning \& Data Mining & Advanced Internet Protocol \\
%& & Computer Networks	& Introduction to Operational Research \\
%& & Communication System and techniques	& Introduction to Internet Protocol \\
%& & Wireless Communications Systems	& Digital Signal Processing \\
%& & Object Oriented Programming \\
%%\pagebreak
%\end{tabular}
\section{{Skills}}
\begin{itemize} \itemsep -2pt
\item Programming: C/C++, Java, Python
%\item Network Stack: TCP/IP, HTTP, UDP, L2
\item Software Packages: MATLAB, R, Android SDK, Pyretic, POX, Kinetic
%\item Platforms: Windows, Linux, CUDA
\end{itemize}
%\section{{References}}
%Available upon requests.
\end{resume}
\end{document}